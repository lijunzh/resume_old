%!TEX TS-program = xelatex
%!TEX encoding = UTF-8 Unicode
% Awesome CV LaTeX Template for CV/Resume
%
% This template has been downloaded from:
% https://github.com/posquit0/Awesome-CV
%
% Author:
% Claud D. Park <posquit0.bj@gmail.com>
% http://www.posquit0.com
%
% Template license:
% CC BY-SA 4.0 (https://creativecommons.org/licenses/by-sa/4.0/)
%


%-------------------------------------------------------------------------------
% CONFIGURATIONS
%-------------------------------------------------------------------------------
% A4 paper size by default, use 'letterpaper' for US letter
\documentclass[11pt, a4paper]{awesome-cv}

% Configure page margins with geometry
\geometry{left=1.4cm, top=.5cm, right=1.4cm, bottom=1.8cm, footskip=.5cm}

% Specify the location of the included fonts
\fontdir[fonts/]

% Color for highlights
% Awesome Colors: awesome-emerald, awesome-skyblue, awesome-red, awesome-pink, awesome-orange
%                 awesome-nephritis, awesome-concrete, awesome-darknight
\colorlet{awesome}{awesome-red}
% Uncomment if you would like to specify your own color
% \definecolor{awesome}{HTML}{CA63A8}

% Colors for text
% Uncomment if you would like to specify your own color
% \definecolor{darktext}{HTML}{414141}
% \definecolor{text}{HTML}{333333}
% \definecolor{graytext}{HTML}{5D5D5D}
% \definecolor{lighttext}{HTML}{999999}

% Set false if you don't want to highlight section with awesome color
\setbool{acvSectionColorHighlight}{true}

% If you would like to change the social information separator from a pipe (|) to something else
\renewcommand{\acvHeaderSocialSep}{\quad\textbar\quad}

% External packages
\usepackage[
	backend=biber,
	style=ieee,
	sorting=ydnt,
	defernumbers=true,
	refsection=section
	]{biblatex}
\addbibresource{lijun.bib}
\AtBeginBibliography{\footnotesize}

% prevent certain fields in references from printing in bibliography
\AtEveryBibitem{\clearfield{issn}}
\AtEveryBibitem{\clearlist{issn}}

\AtEveryBibitem{\clearfield{language}}
\AtEveryBibitem{\clearlist{language}}

\AtEveryBibitem{\clearfield{doi}}
\AtEveryBibitem{\clearlist{doi}}

\AtEveryBibitem{\clearfield{url}}
\AtEveryBibitem{\clearlist{url}}

\AtEveryBibitem{\clearfield{eprint}}
\AtEveryBibitem{\clearlist{eprint}}

\AtEveryBibitem{\clearfield{note}}
\AtEveryBibitem{\clearlist{note}}

\AtEveryBibitem{%
    \ifentrytype{online}
    {}
    {\clearfield{urlyear}\clearfield{urlmonth}\clearfield{urlday}}}


%-------------------------------------------------------------------------------
%	PERSONAL INFORMATION
%	Comment any of the lines below if they are not required
%-------------------------------------------------------------------------------
% Available options: circle|rectangle,edge/noedge,left/right
\photo[circle, edge, left]{lijun}
\name{Lijun}{Zhu}
\position{Research Assistant{\enskip\cdotp\enskip}Signal/Data Analysis}
\address{2626 Staunton Lane, Duluth, GA 30096}

\mobile{(+1)404-545-2619}
\email{lijun.zhu@gatech.edu}
\homepage{www.lijunzhu.info}
\github{lijunzh}
\linkedin{lijunzhugt}
% \gitlab{gitlab-id}
% \stackoverflow{SO-id}{SO-name}
% \twitter{@twit}
% \skype{skype-id}
% \reddit{reddit-id}
% \extrainfo{extra information}

%\quote{``Be the change that you want to see in the world."}

%-------------------------------------------------------------------------------
\begin{document}

% Print the header with above personal information
% Give optional argument to change alignment(C: center, L: left, R: right)
\makecvheader[R]

% Print the footer with 3 arguments(<left>, <center>, <right>)
% Leave any of these blank if they are not needed
\makecvfooter
  {\today}
  {Lijun Zhu~~~·~~~Résumé}
  {\thepage}


%-------------------------------------------------------------------------------
%	CV/RESUME CONTENT
%	Each section is imported separately, open each file in turn to modify content
%-------------------------------------------------------------------------------
%-------------------------------------------------------------------------------
%	SECTION TITLE
%-------------------------------------------------------------------------------
\cvsection{Summary}


%-------------------------------------------------------------------------------
%	CONTENT
%-------------------------------------------------------------------------------
\begin{cvparagraph}

%---------------------------------------------------------
PhD student majored in Electrical Engineering working on array processing and numerical modeling of wave propagation. Passionate about music and audio system. Co-author of FDTD simulation tool \textit{S3I} and contributing to \textit{Obspy}, the open-source seismic signal processing tools. Playing clarinet for over 10 years with five years orchestra experience.
\end{cvparagraph}

%-------------------------------------------------------------------------------
%	SECTION TITLE
%-------------------------------------------------------------------------------
\cvsection{Education}


%-------------------------------------------------------------------------------
%	CONTENT
%-------------------------------------------------------------------------------
\begin{cventries}

%---------------------------------------------------------
  \cventry
    {Ph.D. student in Electrical Engineering} % Degree
    {Georgia Institute of Technology} % Institution
    {Atlanta, GA} % Location
    {Aug. 2014 - PRESENT} % Date(s)
    {
      \begin{cvitems} % Description(s) bullet points
        \item {Research topic: detection and estimation through signal processing, statistical, and machine learning tools.}
        \item {Advisor: Professor James H. McClellan.}
        \item {Expected graduated in 2019.}
      \end{cvitems}
    }
    
  \cventry
    {B.Sc. in Electrical Engineering} % Degree
    {Georgia Institute of Technology} % Institution
    {Atlanta, GA} % Location
    {Aug. 2009 - May. 2013} % Date(s)
    {
      \begin{cvitems} % Description(s) bullet points
%        \item {Focused on Signal Processing.}
        \item {Developed and maintained online tutorial system for DSP undergraduate courses.}
        \item {Designed and tested keyword spotting algorithm for always-on voice recognition system.}
        \item {Built peripheral circuits and wrote programs on Texas Instruments MSP430 chip.}
        \item {Graduated with Highest Honor (GPA = 3.96/4.00).}
      \end{cvitems}
    }

%---------------------------------------------------------
\end{cventries}

%-------------------------------------------------------------------------------
%	SECTION TITLE
%-------------------------------------------------------------------------------
\cvsection{Skills}


%-------------------------------------------------------------------------------
%	CONTENT
%-------------------------------------------------------------------------------
\begin{cvskills}

%---------------------------------------------------------
  \cvskill
    {Coding} % Category
    {Python, C/C++, MATLAB\textregistered, LaTeX, Bash, Assembly/VHDL, Verilog, Perl, PHP, HTML/CSS, SQL} % Skills

%---------------------------------------------------------
  \cvskill
    {Software} % Category
    {AWS, NumPy/SciPy, scikit-learn, PyTorch, TensorFlow, Pandas, OpenCV, awk/sed, GNU Parallel, ssh/scp} % Skills
    
%---------------------------------------------------------
  % \cvskill
  %   {Hardware} % Category
  %   {NI Labview/DAQ, acoustic measurement, soldering, oscilloscope, logic analyzer} % Skills

%---------------------------------------------------------
\end{cvskills}

%-------------------------------------------------------------------------------
%	SECTION TITLE
%-------------------------------------------------------------------------------
\cvsection{Experience}


%-------------------------------------------------------------------------------
%	CONTENT
%-------------------------------------------------------------------------------
\begin{cventries}

%---------------------------------------------------------
  \cventry
    {Research Assistant} % Job title
    {Georgia Institute of Technology} % Organization
    {Atlanta, GA} % Location
    {Sep. 2014 - PRESENT} % Date(s)
    {
      \begin{cvitems} % Description(s) of tasks/responsibilities
        \item {Prepare and process large data using bash/awk script on Linux/Unix servers.}        
        \item {Design, prototype, and test machine learning algorithms using \textit{Python} on large-scale dataset with tools like PyTorch and Tensorflow.}
        \item {Maintain and upgrade Linux HPC cluster and storage system.}
        \item {Develop and support in-house numerical (FD) simulation tools for elastic wave propagation in complex medium.}
%        \item {Document and organize results in reports, user manuals, and research papers.}
      \end{cvitems}
    }

%---------------------------------------------------------
\cventry
{System Engineer} % Job title
{Texas Instruments} % Organization
{Dallas, TX} % Location
{May. 2018 - July. 2018} % Date(s)
{
    \begin{cvitems} % Description(s) of tasks/responsibilities
        \item {Implement and benchmark convolutional neural network (CNN) performance for vision tasks.}
        \item {Design and test quantization of CNNs with and without fine-tuning.}
        \item {Investigate possibility of novel quantization scheme for low-bit fixed-point arethmetics.}
    \end{cvitems}
}

%---------------------------------------------------------
  \cventry
    {Research Intern} % Job title
    {Houston Research Center, Aramco Service Company} % Organization
    {Houston, TX} % Location
    {Aug. 2015 - Nov. 2015} % Date(s)
    {
      \begin{cvitems} % Description(s) of tasks/responsibilities
        \item {Wrote \textit{Python/MATLAB} tools for organizing and processing large-scale dataset (> 1TB).}
        \item {Processed land acquisition data searching for small events in the noisy environment.}
        \item {Tested machine learning algorithms for dimension reduction, image segmentation and object tracking on spectrogram domain.}
%        \item {Built tools to enhance signal quality and cancel noise using dictionary learning backed algorithms.}
      \end{cvitems}
    }

%---------------------------------------------------------
  \cventry
    {Research Intern} % Job title
    {Microsoft Research} % Organization
    {Redmond, WA} % Location
    {May. 2014 - Aug. 2014} % Date(s)
    {
      \begin{cvitems} % Description(s) of tasks/responsibilities
        \item {Wrote numerical simulation tools for ultrasonic wave propagation in \textit{C++} with a \textit{MATLAB} interface.}
        \item {Conducted acoustic measurements in anechoic chamber testing prototype products.}
        \item {Documented progress and results in published research papers.}
      \end{cvitems}
    }

%---------------------------------------------------------
  \cventry
    {Research Co-op} % Job title
    {Bose Coporation} % Organization
    {Framingham, MA} % Location
    {Jan. - May., Aug. - Dec. 2012} % Date(s)
    {
      \begin{cvitems} % Description(s) of tasks/responsibilities
        \item {Worked with marketing team in identifying customer's requirements and make product definition.}
%        \item {Worked with transducer team to designed and simulated transducer array using their newly developed waveguide and subwoofer.}
%        \item {Built prototype speaker system on wood frame and developed noise reduction circuit for power amplifier.}
%        \item {Measured and evaluated acoustic system in both anechoic chamber and living room setup (both objective and subject evaluations).}
        \item {Led the product prototyping in early stage and make demonstration to executives.}
%        \item {Design went into development and eventually became a mass market product.}
        \item {Updated \textit{MATLAB} and \textit{Perl} script to automate testing procedure.}
        \item {Assisted adaptive microphone array design for conference setup.}
      \end{cvitems}
    }

%---------------------------------------------------------
\end{cventries}

\vspace*{1cm}
%-------------------------------------------------------------------------------
%	SECTION TITLE
%-------------------------------------------------------------------------------
\cvsection{Services}


%-------------------------------------------------------------------------------
%	CONTENT
%-------------------------------------------------------------------------------
\begin{cventries}

%---------------------------------------------------------
  \cventry
    {Vice President \& President} % Affiliation/role
    {SEG Student Chapter in Georgia Tech} % Organization/group
    {Atlanta, GA} % Location
    {Jun. 2016 - PRESENT} % Date(s)
    {
      \begin{cvitems} % Description(s) of experience/contributions/knowledge
        \item {Organized annual meeting and community services in 2016.}
        \item {Led the development of student chapter website redesign in 2017.}
      \end{cvitems}
    }
%---------------------------------------------------------

  \cventry
    {Secretary} % Affiliation/role
    {IEEE Student Chapter in Georgia Tech} % Organization/group
    {Atlanta, GA} % Location
    {Feb. 2011 - Oct. 2011} % Date(s)
    {
      \begin{cvitems} % Description(s) of experience/contributions/knowledge
        \item {Organized meetings and community services.}
        \item {Kept meeting minutes and updated chapter website.}
      \end{cvitems}
    }
%---------------------------------------------------------
\end{cventries}

%-------------------------------------------------------------------------------
%	SECTION TITLE
%-------------------------------------------------------------------------------
\cvsection{Honors \& Awards}


%-------------------------------------------------------------------------------
%	SUBSECTION TITLE
%-------------------------------------------------------------------------------
\cvsubsection{International}


%-------------------------------------------------------------------------------
%	CONTENT
%-------------------------------------------------------------------------------
\begin{cvhonors}

%---------------------------------------------------------
  \cvhonor
    {Finalist} % Award
    {DEFCON 25th CTF Hacking Competition World Final} % Event
    {Las Vegas, U.S.A} % Location
    {2017} % Date(s)

%---------------------------------------------------------
  \cvhonor
    {Finalist} % Award
    {DEFCON 22nd CTF Hacking Competition World Final} % Event
    {Las Vegas, U.S.A} % Location
    {2014} % Date(s)

%---------------------------------------------------------
  \cvhonor
    {Finalist} % Award
    {DEFCON 21st CTF Hacking Competition World Final} % Event
    {Las Vegas, U.S.A} % Location
    {2013} % Date(s)

%---------------------------------------------------------
  \cvhonor
    {Finalist} % Award
    {DEFCON 19th CTF Hacking Competition World Final} % Event
    {Las Vegas, U.S.A} % Location
    {2011} % Date(s)

%---------------------------------------------------------
  \cvhonor
    {6th Place} % Award
    {SECUINSIDE Hacking Competition World Final} % Event
    {Seoul, S.Korea} % Location
    {2012} % Date(s)

%---------------------------------------------------------
\end{cvhonors}


%-------------------------------------------------------------------------------
%	SUBSECTION TITLE
%-------------------------------------------------------------------------------
\cvsubsection{Domestic}


%-------------------------------------------------------------------------------
%	CONTENT
%-------------------------------------------------------------------------------
\begin{cvhonors}

%---------------------------------------------------------
  \cvhonor
    {3rd Place} % Award
    {WITHCON Hacking Competition Final} % Event
    {Seoul, S.Korea} % Location
    {2015} % Date(s)

%---------------------------------------------------------
  \cvhonor
    {Silver Prize} % Award
    {KISA HDCON Hacking Competition Final} % Event
    {Seoul, S.Korea} % Location
    {2013} % Date(s)

%---------------------------------------------------------
\end{cvhonors}

%\nobibliography{publication/lijun}
%\bibliographystyle{unsrt}
%\vspace*{-6cm}
%-------------------------------------------------------------------------------
%	SECTION TITLE
%-------------------------------------------------------------------------------
\cvsection{Publication}
    \nocite{zhu2019deep,li2018high,liu2017microseismic,zhu2017multi}
    \nocite{chuang2019deep,li2018lightweight,li2018transfer,zhu2018deep,li2018high,zhu2018event,zhu2017classification,zhu2017machine,zhu2017weighted,li2017high,zhu2016automatic,liu2016microseismic,zhu20153d,zhu2015full}
    \vspace*{-5mm}
    \printbibliography[type=article, title={\small \textbf{Referenced Journals}}]
    \vspace{-5mm}
    \printbibliography[type=inproceedings, title={\small \textbf{Conference Proceedings}}]




%-------------------------------------------------------------------------------
%	CONTENT
%-------------------------------------------------------------------------------
%\begin{cventries}
%%---------------------------------------------------------
%  \cventry
%    {} % Affiliation/role
%    {Referenced Journals} % Organization/group
%    {} % Location
%    {} % Date(s)
%    {
%      \begin{cvitems} % Description(s) of experience/contributions/knowledge
%        \item {\bibentry{Zhu2017d}}
%        \item {\bibentry{Liu2017}}
%      \end{cvitems}
%    }
%
%  \vspace*{5mm}
%
%  \cventry
%    {}
%    {Conference Abstracts}
%    {}
%    {}
%    {
%      \begin{cvitems}
%          \item {\bibentry{Zhu2017c}}
%          \item {\bibentry{Zhu2017b}}
%          \item {\bibentry{Zhu2017a}}
%          \item {\bibentry{Li2017}}
%          \item {\bibentry{Zhu2016}}
%          \item {\bibentry{Liu2016}}
%          \item {\bibentry{Zhu2015b}}
%          \item {\bibentry{Zhu2015a}}
%      \end{cvitems}
%    }
%%---------------------------------------------------------
%\end{cventries}




%-------------------------------------------------------------------------------
\end{document}